% Chapter Template

\chapter{Conclusiones} % Main chapter title

\label{Chapter5} % Change X to a consecutive number; for referencing this chapter elsewhere, use \ref{ChapterX}


%----------------------------------------------------------------------------------------

%----------------------------------------------------------------------------------------
%	SECTION 1
%----------------------------------------------------------------------------------------

\section{Conclusiones generales }

En el presente Trabajo Final se ha logrado obtener un entorno de depuración para la plataforma CIAABOT. El entorno de depuración \emph{ CIAABOT Debug} otorga una interfaz simple e intuitiva en el uso de herramientas de depuración, utilizando la plataforma EDU-CIAA-NXP, y logrando de esta manera los objetivos planteados.

En el desarrollo de este Trabajo Final se aplicaron los conocimientos adquiridos en la Carrera de Especialización en Sistemas Embebidos, en especial las asignaturas:

\begin{itemize}
	\item Ingeniería de software en sistemas embebidos. Se utilizaron técnicas y metodologías de trabajo de la ingeniería del software. Se usaron los conocimientos de manejo de repositorio Git como sistema de control de versiones y se aplicaron metodologías de ensayo de software.
	\item Gestión de Proyectos en Ingeniería. Plan de Proyecto para organizar el Trabajo Final, permitió desde el inicio de la carrera tener una visión de todas las tareas a realizar y su estimación en tiempos.
	\item Protocolos de Comunicación. Se aplicaron los conocimientos obtenidos del protocolo Firmata.
	\item Sistemas Operativos de Propósito General. Se usaron los conocimientos sobre Linux, usando las herramientas que provee para usarse en las pruebas generales del software creado.
\end{itemize}



 
% Aportes del trabajo realizado. Que se logro realizar. 
% Evaluar el cumplimiento de los objetivos y requerimientos.
% Que se logoro realizar (no planeado inicialmente).
% Cómo se realizo:
% Metodología aplicada
% Desafios a lo largo del proyecto.
% Que conocimientos puse en práctica.
% Conocimientos adquiridos.
% Reflexion final.

El entorno de \emph{CIAABOT debug} le da la capacidad de depuración al proyecto CIAABOT, por lo tanto tambien aporta al Proyecto CIAA, y a la comunidad de sistemas embebidos en general.

Se concluye que los objetivos del trabajo han sido cumplidos satisfactoriamente, se logró un mejor conocimiento del hardware y el desarrollo del mismo ha favorecido en
gran medida a la formación profesional del autor.


%----------------------------------------------------------------------------------------
%	SECTION 2
%----------------------------------------------------------------------------------------
\section{Próximos pasos}

El desarrollo del entorno de \emph{debug} es amplio, debe conformar más herramientas y utilidades. En un futuro se cuenta lograr con más características, tales como: 

\begin{itemize}
	\item Capacidad de evaluación de expresiones.
	\item Ver stack de llamadas a funciones.
	\item Consola de comandos del \emph{debug}, en este caso GDB.
\end{itemize}
