% Chapter Template

\chapter{Conclusiones} % Main chapter title

\label{Chapter5} % Change X to a consecutive number; for referencing this chapter elsewhere, use \ref{ChapterX}


%----------------------------------------------------------------------------------------

%----------------------------------------------------------------------------------------
%	SECTION 1
%----------------------------------------------------------------------------------------

\section{Conclusiones generales }


La implementan del entorno de \emph{debug} para CIAABOT tuvo como objetivo el mejorar la plataforma CIAABOT que se encuentra productivo hoy en día, y ofrece a los usuarios una manera didáctica de aprendizaje. \emph{CIAABOT Debug} pretendió desde un comienzo el seguir aportando a sus usuarios herramientas que mejoren sus conocimientos.

La experiencia de la reunion con los usuarios finales hizo posible el surgimiento de nuevos desafíos a partir de las necesidades que fueron expuestas.

 

%----------------------------------------------------------------------------------------
%	SECTION 2
%----------------------------------------------------------------------------------------
\section{Próximos pasos}

El desarrollo del debug es amplio, conforma más herramientas. En un futuro se cuenta 
lograr con más características, tales como: 

\begin{itemize}
	\item Ver expresiones.
	\item Ver stack de llamadas a funciones.
	\item Consola de comandos del debugger, en este caso GDB).
\end{itemize}
